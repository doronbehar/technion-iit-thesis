% General-purpose definitions and inclusions
% you are using in any document 
% (regardless of its class and style files used),
% e.g. package uses:

%\usepackage{xspace}

% and macros/command defintions:

%\newcommand{\complexityclass}[1]{{\bf #1}\xspace}
%\newcommand{\NPTIME}{\complexityclass{NP}}

% For this template, we'll only have one single command,
% necessary for including graphics...
\usepackage{graphicx}% http://ctan.org/pkg/graphicx
\usepackage{amssymb}
\usepackage{amsmath}
\usepackage{tabularx}
\usepackage[version=4]{mhchem}
\hypersetup{colorlinks = true,
	citecolor = gray,
	linkcolor = red,
	citecolor = green,
	filecolor = magenta,
	urlcolor = cyan
}
\usepackage{longtable}
%\usepackage{listings}
%% From https://www.overleaf.com/learn/latex/Code_listing
%\usepackage{xcolor}
%
%\definecolor{codegreen}{rgb}{0,0.6,0}
%\definecolor{codegray}{rgb}{0.5,0.5,0.5}
%\definecolor{codepurple}{rgb}{0.58,0,0.82}
%\definecolor{backcolour}{rgb}{0.95,0.95,0.92}
%
%\lstdefinestyle{mystyle}{
%    backgroundcolor=\color{backcolour},   
%    commentstyle=\color{codegreen},
%    keywordstyle=\color{magenta},
%    numberstyle=\tiny\color{codegray},
%    stringstyle=\color{codepurple},
%    basicstyle=\ttfamily\footnotesize,
%    breakatwhitespace=false,         
%    breaklines=true,                 
%    captionpos=b,                    
%    keepspaces=true,                 
%    numbers=left,                    
%    numbersep=5pt,                  
%    showspaces=false,                
%    showstringspaces=false,
%    showtabs=false,                  
%    tabsize=2
%}
%
%\lstset{style=mystyle}

