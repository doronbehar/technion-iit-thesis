\chapter{Introduction}\label{chap:intro}

\section{What are Chiral Molecules?}\label{what-are-chiral-molecules}

\section{Our Experimental Scheme}\label{our-experimental-scheme}

\section{All Trapping \& Cooling Parameters}\label{all-trapping-cooling-parameters}

The experimental scheme described above justifies optimizing the cooling
process, and hence invite the main effort and research topic of this
thesis - simulations of the trapping and (sympathetic) cooling our
molecule. The simulations were performed by LAMMPS , which is capable of
simulating numerous classical models relevant in many contexts, but in
our case, I chose a simple model of charged point particles with only
long-range coulomb interactions, and an electrical field resembling the
field generated by our ion trap, along with a velocity dependent force
imitating the LASER cooling. Justifying the resemblance of the trapping
and cooling parameters in real life to the simulation's code is detailed
in the subsections below, along with the technical parameters subsection
following.

\subsection{Trapping}\label{trapping}

Our ion's trap design is based upon Jila's..

\subsection{Cooling}\label{cooling}

Detuning and intensity were barely studied in previous research\ldots,
cooler (Yb/Ca/Be)

\subsection{Technical}\label{technical}

Cooling, stabilizing time, rf divisor

\section{Output Result Types}\label{output-result-types}
