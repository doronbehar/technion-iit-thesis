\chapter{Introduction}\label{chap:intro}

% Slightly copy from Itay Erez's thesis? Or simply cite it?
% Explain in more detail about our candidates and from there talk about IVR


\section{What are Chiral Molecules?}\label{what-are-chiral-molecules}

\section{Our Experimental Scheme}\label{our-experimental-scheme}

% Explain about our the general scheme, or cite something?
% Explain about our ion trap in details, especially details relevant to the velocity / kinetic energy resolution required and hence the maximal temperatures required.
% Show the level diagram for Yb+ and from there justify the need for a 935nm LASER 
% Comparison to typical molecular ion trapping setups

\section{All Trapping \& Cooling Parameters}\label{all-trapping-cooling-parameters}

The experimental scheme described above justifies optimizing the cooling
process, and hence invite the main effort and research topic of this
thesis - simulations of the trapping and (sympathetic) cooling our
molecule. The simulations were performed by LAMMPS , which is capable of
simulating numerous classical models relevant in many contexts, but in
our case, I chose a simple model of charged point particles with only
long-range coulomb interactions, and an electrical field resembling the
field generated by our ion trap, along with a velocity dependent force
imitating the LASER cooling. Justifying the resemblance of the trapping
and cooling parameters in real life to the simulation's code is detailed
in the subsections below, along with the technical parameters subsection
following.

\subsection{Trapping}\label{trapping}

% Frequencies..., pseudo, Mathieu...

Our ion's trap design is based upon Jila's..

\subsection{Cooling}\label{cooling}

Detuning and intensity were barely studied in previous research\ldots,
cooler (Yb/Ca/Be)

% Mention the relation of intensity to mW/cm^2
% Mention theory best explained at Dan Steck's stuff.
% Explain how

\subsection{Technical}\label{technical}

Cooling, stabilizing time, rf divisor

% Mention the challenge of initializing the system in a thermodynamic stable condition
% Put all the onenote's technical challenges related content here

\section{Output Result Types}\label{output-result-types}

% What kind of scalar results from the measurements are of interest to us? T_final, T_middle etc, mention also the cloud sizes and the relation to the experiment's measurement methods
% Naturally, explain the behavior of the cooling regimes etc.

