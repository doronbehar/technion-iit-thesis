% This file contains the abstract part of your thesis - in English and
% in Hebrew (within \abstractEnglish and \abstractHebrew respectively).
%
% Notes:
% - This file uses the UTF-8 character set encoding for the Hebrew
%   text not to get garbled. Keep it that way.
% - Assuming your thesis is mainly in English, Graduate School 
%   regulations mandate the following lengths for the abstracts:
%
%      Language    Min. Length   Max. Length
%     ---------------------------------------
%      English       200 words     500 words
%      Hebrew        500 words   2,000 words
%
%   so that the Hebrew abstract typically has some content from
%   the English introduction and an overview of the results, not
%   present in the English; it is not just a translation.

\abstractEnglish{

The weak force in the standard model is considered to be responsible for
energy shifts in energy levels and can be isolated when comparing the
spectrum of left and right handed chiral molecules. Our group's goal is
to measure this shift, which has never been measured directly with
spectroscopic methods.

Standard model numerical calculations predict this weak force shift, also
referred to as parity violation (PV) shift to be \(\Delta_{PV} = 1.8Hz\) at
best, in the \ce{C-H} wag vibration of \ce{CHDBrI^+}\cite{IVR_article}, while
the vibration itself is at the (IR) frequency of \(\nu_v = 33 (\sim 9 \mu m)\)!
This \ce{C-H} wag vibrational frequency of our candidate enantiomers, as well
as for other molecules we considered, is roughly \(23\pm 4 \,\mathrm{cm^{-1}}\)
times higher then the lowest frequency vibration, with 7 additional vibrational
frequencies in between (9 altogether). This implies that the lifetime of the
excited \ce{C-H} wag state of any of our most promising candidate molecules
will be limited by intra-molecular vibration redistribution (IVR). An estimate
of the extent of this limitation was calculated numerically and it's results
are described in detail in this thesis. In \cite{IVR_article} you can find a
brief summary of the results.

Our \(\Delta_{PV}\) measurement plan, broadly speaking consists of (1)
initializing our candidate molecule (\ce{CHDBrI}), (2) ionizing it, (3)
sympathetically cool it with an easily LASER cooled metal (\ce{Yb^+} in our
current attempts), (4) employ a 3 wave mixing technique to create an
enantiomer-selective quantum interference, which separates enantiomers
into different quantum state.

Sympathetically cooling a molecule is well known, already accomplished achievement, % Cite a few examples
. However, \emph{our} measurement plan is substantially different from the typical
experimental schemes employed in other experiments in which molecules
are sympathetically cooled, because the molecule initialization is not
trivial. In our experiment's plan the molecule is trapped for periods
\(<100ms\), and hence the cooling must be as effective as possible. Not
only that, our trap design is unique in the field of ion trapping and
cooling, and it provides us a flexible range of secular frequencies that
are not used often. A brief comparison to typical trapping experiments
is laid out in the introduction. This non-typical setup and requirements
invites a thorough simulations based research on the parameter space of
the trapping \& LASER cooling, which is the main topic of this thesis.

Finally, a few details of a \(935\,\mathrm{nm}\) LASER diode setup are
shown at the end of the thesis. This LASER is required for optical
pumping $1\%$ of the \ce{Yb+} ions that are supposed to sympathetically cool
our \ce{CHDBrI+}.

} % end of English abstract


\abstractHebrew{

% Note that certain commands don't work that well in Hebrew "mode".
% If this happens to you, try wrapping the command within a
% \textenglish{ } - that may (or may not) help.

כאן יבוא תקציר מורחב בעברית (כאשר שפת החיבור העיקרית היא אנגלית). היקף התקציר יהיה \textenglish{1000-2000} מילים. התקציר יהווה שלמות בפני עצמו ויהיה מובן לקורא בעל ידיעות כלליות בנושא.

בית הספר ללימודי מוסמכים מנחה מספר הנחיות לגבי התקציר בעברית:
\begin{itemize}
\item על התקציר להיכתב במשפטים מקושרים שלמים.
\item בדרך-כלל אין לציין בתקציר מקורות ספרותיים וציטוטים.
\item אין להתייחס למספר של פרק, סעיף, נוסחה, ציור או טבלה שבגוף החיבור, ואין להשתמש בקיצורים, סמלים ומונחים לא מקובלים, אלא אם יש בתקציר די מקום לזיהויים.
\end{itemize}

לעתים יש בכל-זאת יש צורך לכלול פקודה הכוללת קישור פנימי או חיצוני בתוך התקציר העברי; במצבים כאלו כדאי דרך-כלל לעטוף את הפקודה היוצרת את הקישור בתוך פקודת \textenglish{\texttt{\textbackslash{}textenglish\{\}}} כדי למנוע כל מיני פורענויות בלתי-רצויות, כגון כישלון בהידור קובץ ה-\textenglish{PDF} או שימוש בגופן העברי באופן אשר עלול שלא להנעים לעין. לדוגמה: נניח שיש לנו צורך לצטט מקור ביבליוגרפי. אם נעשה זאת סתם-כך: \textenglish{\texttt{\textbackslash{}cite\{Hoeffding\}}}, נקבל: \cite{Hoeffding}; אם נעטוף את פקודת הציטוט, כך: \textenglish{\texttt{\textbackslash{}textenglish\{\textbackslash{}cite\{Hoeffding\}\}}}, נקבל \textenglish{\cite{Hoeffding}} (כפי שהציטוטים נראים גם בטקסט באנגלית).

\subsection*{\texthebrew{תת-חלק בתקציר המורחב}}

תוכן מקוצר לגבי נושא מסוים. התייחסות ל\emph{מושג} מסוים שהחיבור בוחן. וכולי וכולי.


\subsection*{\texthebrew{נקודה מעניינת לגבי העמודים בעברית}}

שימו לב כי העמודים בעברית אמורים להיות מיוצרים בסדר ה''הפוך'', הווה אומר העמוד האחרון בקובץ ה-\textenglish{PDF} הוא הכריכה העברית, לפניו השער העברי, ודפי התקציר צריכים להופיע בסדר הפוך (וכן במספור רומי, לפי נהלי הטכניון). כך אם נתבונן במספר שבתחתית עמוד זה \textenglish{---} אשר צריך להיות העמוד הראשון בתקציר-המורחב מבחינת רצף התוכן, והינו העמוד האחרון מבין עמודי התקציר-המורחב אחרון בקובץ ה-\textenglish{PDF} \textenglish{---} נמצא את המספר \textenglish{i} ...

\newpage

... ואילו עמוד זה של התקציר-המורחב בעברית \textenglish{---} שהינו העמוד השני בתקציר-המורחב מבחינת רצף התוכן, ונמצא ראשון בקובץ ה-\textenglish{PDF} \textenglish{---} ממוספר ב-\textenglish{ii}. המטרה במספור בסדר ה"הפוך" היא, שבעת ההדפסה לא יהיה צורך להפוך דפים, לשנות את סדרם וכולי \textenglish{---} רק להדפיס ולכרוך.

} % end of Hebrew abstract
